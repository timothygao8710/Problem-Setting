\documentclass[15pt]{article}

% Language setting
% Replace `english' with e.g. `spanish' to change the document language
\usepackage[english]{babel}

% Set page size and margins
% Replace `letterpaper' with `a4paper' for UK/EU standard size
\usepackage[letterpaper,top=2cm,bottom=2cm,left=3cm,right=3cm,marginparwidth=1.75cm]{geometry}

% Useful packages
\usepackage{amsmath}
\usepackage{graphicx}
\usepackage[colorlinks=true, allcolors=blue]{hyperref}
\usepackage{listings}
\usepackage{color}

\definecolor{dkgreen}{rgb}{0,0.6,0}
\definecolor{gray}{rgb}{0.5,0.5,0.5}
\definecolor{mauve}{rgb}{0.58,0,0.82}

\lstset{frame=tb,
  language=C++,
  aboveskip=3mm,
  belowskip=3mm,
  showstringspaces=false,
  columns=flexible,
  basicstyle={\small\ttfamily},
  numbers=none,
  numberstyle=\tiny\color{gray},
  keywordstyle=\color{blue},
  commentstyle=\color{dkgreen},
  stringstyle=\color{mauve},
  breaklines=true,
  breakatwhitespace=true,
  tabsize=3
}

\title{Highways Analysis}
\author{Timothy Gao}
\date{July 2022}

\begin{document}

\maketitle

To reformulate the problem, we're essentially given a grid with N ($10^9$) rows and M ($2\cdot 10^5$) columns. We are tasked with answering Q ($2\cdot 10^5$) queries, each of which asks if it's possible to travel from cell row $A$, column $B$ to row $C$, column $D$ by staying between rows $A$ and $C$. We are free to move up and down between rows, but not across columns. We are also given $K$ roads, each of which allows us to move freely between any columns in the range $[l,r]$ on row $i$.

\section{Initial Observations}

Let's consider two overlapping roads on the same row, with ranges $[l_i,r_i]$ and $[l_j,r_j]$, $l_i \leq l_j \leq r_i \leq r_j$. From any point on $[l_i, l_j)$, we can move to any point on the non-empty range $[l_j, r_i]$, then to any point on the range $(r_i, r_j]$. This implies that we can treat them as a single range $[l_i, r_j]$, and the same argument can be applied for any overlapping ranges on the same row.

Let's focus on a single query with parameters $A_i, B_i, C_i, D_i$. 
For any two roads on $r_i$ and $r_j$, as long as $A_i \leq r_i \leq C_i$ and $A_i \leq r_j \leq C_i$, we are free to move between the roads wherever they share the same column. With this, we can apply the previous argument and combine every pair of overlapping ranges into a single larger range.

Essentially, we "collapse" all roads with rows between $A_i$ and $C_i$ into a one-dimensional space. With this observation, the query becomes finding whether, after combining all column-wise overlapping roads between rows $A_i$ and $C_i$, we are able to move from column $B_i$ to $D_i$ (i.e., $[B_i, D_i]$ is contained in a larger range after merging roads).

\section{Arriving at the Solution}

The order in which we process the queries don't matter. Let's see how we can use sweep line to answer queries offline, sorting by $max(A_i, C_i)$ for each query $i$.

Instead of actually merging all the roads each query, let's use the properties of sweep line to our advantage. In particular, as we iterate from row $1$ to $N$, let's maintain for each column the most recent (highest-indexed) row that has a road containing the column. In other words, we maintain an $V$ array of size $M$, and for each row $i \in 1 ... N$, we iterate through each road $R$ on row $i$ and set $V[R_l], V[R_l + 1] ... V[R_r - 1], V[R_r]$ equal to $i$. Now, answering each query $j$ on the $max(A_j, C_j)$th row becomes intuitive - we simply query whether $min(V[min(B_j, D_j)] ... V[max(B_j, D_j)]) \geq min(A_j, C_j)$. With this formulation, notice that we don't actually need to iterate through all rows $1...N$. We only care about the rows that appear in either a road or a query, so we can perform coordinate compression.

Updates and queries can accomplished in $\mathcal{O} (K log(M))$ and $ \mathcal{O} (K log(M))$ respectively with a lazy segment tree. The initial sorting takes $\mathcal{O} (M log(M))$, leading to a final time complexity of $\mathcal{O} ((N + Q + M) \cdot log(M))$.

\section{Code}
\begin{lstlisting}

//@timothyg

#include <bits/stdc++.h>
using namespace std;

typedef long long ll;
typedef pair<int, int> pii;

#define pb push_back

#define f first
#define s second

template <typename T>
struct segtree
{
	T none, val, lazy;
	int gL, gR, mid;
	segtree<T>* left, * right;

	T comb(T& l, T& r)
	{
		return min(l, r);
	}

	//In this code, lazy represents the value only to be propagated(this->value already updated)
	void compose(segtree<T>* tree, T v)
	{
		tree->lazy = max(v, tree->lazy);
		tree->val = max(tree->val, tree->lazy);
	}

	void push()
	{
		if (gL != gR)
		{
			compose(left, lazy);
			compose(right, lazy);
		}
	}

	segtree(int l, int r)
	{ //modify arr type
		none = INT_MAX;
		lazy = -1;
		gL = l, gR = r, mid = (gL + gR) / 2;
		if (l == r)
		{
			val = -1;
		}
		else
		{
			left = new segtree<T>(l, mid);
			right = new segtree<T>(mid + 1, r);
			val = comb(
				left->val, right->val);
		}
	}

	T query(int l, int r)
	{
		if (gL > r || gR < l)
		{
			return none;
		}

		if (gL == l && gR == r)
		{
			return val;
		}
		push();
		if (val == -1)
		{
			val = -1;
		}
		T a = left->query(l, min(r, mid));
		T b = right->query(max(l, mid + 1), r);
		return comb(a, b);
	}

	void update(int l, int r, T updlazy)
	{
		if (gL > r || gR < l) {
			return;
		}

		if (gL == l && gR == r) {
			compose(this, updlazy);
		}
		else {
			push();
			left->update(l, min(r, mid), updlazy);
			right->update(max(l, mid + 1), r, updlazy);
			val = comb(left->val, right->val);
		}
	}
};

int main() {
	ios_base::sync_with_stdio(false); cin.tie(0);
	int N, M, K, Q; cin >> N >> M >> K >> Q;
	vector<pii> roads(K);
	vector<array<int, 3>> bps;
	for (int i = 0; i < K; i++) {
		int lane, l, r; cin >> l >> r >> lane;
		bps.pb({ lane, 0, i });
		roads[i] = pii(l, r);
	}
	vector<array<int, 4>> queries(Q);
	vector<int> res(Q);
	for (int i = 0; i < Q; i++) {
		int A, B, C, D; cin >> A >> B >> C >> D;
		if (D < B) {
			swap(A, C); swap(B, D);
		}
		bps.pb({ D, 1, i });
		queries[i] = { A, B, C, D };
	}
	sort(bps.begin(), bps.end());
	segtree<int> sgt(0, 2*M + 2);

	for (auto i : bps) {
		if (i[1]) {
			array<int, 4> cur = queries[i[2]];

			//collapse roads
			int mn = sgt.query(min(2*cur[0], 2*cur[2]), max(2*cur[0], 2*cur[2]));
			if (mn >= cur[1] || cur[0] == cur[2]) {
				res[i[2]] = 1;
			}
		}
		else {
			sgt.update(2*roads[i[2]].f, 2*roads[i[2]].s, i[0]);
		}
	}
	for (int i : res) {
		cout << (i ? "YES" : "NO") << '\n';
	}
	return 0;
}
\end{lstlisting}
\end{document}